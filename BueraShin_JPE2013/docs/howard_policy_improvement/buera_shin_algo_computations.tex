\documentclass{article}
\usepackage[utf8]{inputenc}
\usepackage{amsmath}
\usepackage{amssymb}
\usepackage{geometry}
\usepackage{bm}
\usepackage{xcolor}
\geometry{a4paper, margin=1in}

\title{Replication of Buera \& Shin (2013) \\ Implementation with Howard Policy Improvement and Histograms}
\author{Piero De Dominicis \\ Bocconi University}
\date{\today}

\begin{document}

\maketitle

\section{The Economic Environment}
The economy consists of a continuum of infinitely lived agents of measure one. Agents are heterogeneous in their asset holdings $a$ and entrepreneurial ability $z$.

\subsection{Preferences and Shocks}
Agents maximize expected lifetime utility:
\begin{equation}
    U_0 = \mathbb{E}_0 \sum_{t=0}^{\infty} \beta^t \frac{c_t^{1-\sigma}-1}{1-\sigma}
\end{equation}
Entrepreneurial ability $z$ follows a reset process. In each period, with probability $\psi$, the ability remains constant: $z_{t+1} = z_t$. With probability $1-\psi$, a new ability is drawn $z' \sim \pi(z)$ from a Pareto distribution:
\begin{equation}
    P(Z \leq z) = 1 - z^{-\eta}, \quad z \in [1, \infty)
\end{equation}
To simulate distortions in the pre-reform state, agents are subject to idiosyncratic output wedges $\tau \in \{\tau^+, \tau^-\}$. The probability of being in the high-tax state $\tau^+$ is conditional on $z$: $P(\tau = \tau^+ | z) = 1 - e^{-q z}$.

\section{Firm Decisions and Optimal Production}
Entrepreneurs operate a technology $y = z(k^\alpha l^{1-\alpha})^{1-\nu}$. They face a collateral constraint $k \leq \lambda a$.

\subsection{Analytical Solutions for Firm Decisions}
Given $(w, r, a, z)$, the entrepreneur's problem is solved in two stages. 
1. \textbf{Conditional Labor Demand}: For any $k$, the optimal labor $l^*(k, z)$ solves the first-order condition $\partial \pi / \partial l = 0$:
\begin{equation}
    l^*(k, z) = \left[ \frac{(1-\nu)(1-\alpha)z}{w} \right]^{\frac{1}{1-(1-\alpha)(1-\nu)}} k^{\frac{\alpha(1-\nu)}{1-(1-\alpha)(1-\nu)}}
\end{equation}
2. \textbf{Optimal Capital}: Substituting $l^*$ into the profit function yields a concave function $\pi(k)$. The unconstrained optimal capital $k^{unc}$ is found where $\partial \pi / \partial k = 0$. The actual capital choice is capped by the leverage limit:
\begin{equation}
    k^*(a, z) = \min \{ k^{unc}(z, w, r), \lambda a \}
\end{equation}
The total income of an agent is $i(a, z; w, r) = \max \{ \pi(a, z, k^*, l^*), w \} + (1+r)a$.

\section{State Space Discretization}
The state space $\mathcal{A} \times \mathcal{Z}$ is discretized for numerical implementation.

\subsection{Asset Grid}
To capture the high curvature of the value function near the borrowing constraint, a power-spaced grid for $a \in [a_{min}, a_{max}]$ is used:
\begin{equation}
    a_i = a_{min} + (a_{max} - a_{min}) \left( \frac{i}{N_a - 1} \right)^p, \quad i \in \{0, \dots, N_a-1\}
\end{equation}
where $p=2$ provides a dense cluster of points at low asset levels.

\subsection{Ability Grid}
The Pareto distribution $\pi(z)$ is sampled into $N_z$ bins of equal probability. Let $u$ be a uniform random variable. The grid points are midpoints of probability intervals $[F(z_j), F(z_{j+1})]$ where $F$ is the Pareto CDF:
\begin{equation}
    z_j = \left( 1 - \bar{u}_j \right)^{-1/\eta}, \quad \bar{u}_j = \frac{j + 0.5}{N_z} \cdot u_{max}
\end{equation}

\section{Numerical Value Function Iteration}
The value function $V(a, z)$ is solved on the discretized grid using an optimized iterative process.

\subsection{Initialization}
The initial guess $V^{(0)}$ is set to the utility of consuming the current income forever at the lowest grid point, providing a stable starting point for the iteration:
\begin{equation}
    V^{(0)}(a_i, z_j) = \frac{u(i(a_i, z_j) - a_0)}{1-\beta}
\end{equation}

\subsection{Expectation Splitting}
To compute the continuation value $\mathbb{E}[V(a', z') | z]$, the algorithm splits the expectation to exploit the discrete nature of the reset process. Before making decisions, the "reset" expectation $\bar{V}(a')$ is pre-calculated:
\begin{equation}
    \bar{V}(a') = \sum_{j} \pi_j V(a', z_j)
\end{equation}
The specific continuation value for state $(a, z)$ becomes a weighted average of the current ability's value and the reset value:
\begin{equation}
    \text{Cont}(a', z) = \psi V(a', z) + (1-\psi) \bar{V}(a')
\end{equation}

\subsection{Howard Policy Improvement}
The Bellman operator $T(V)$ is computationally expensive due to the maximization step. Howard's improvement reduces the number of maximizations by iterating on a fixed policy $\sigma(a, z)$.
\begin{enumerate}
    \item \textbf{Policy Step}: Solve for $\sigma(a, z) = \arg\max_{a'} \{ u(c) + \beta \text{Cont}(a', z) \}$.
    \item \textbf{Value Step}: For a fixed $\sigma$, iterate $N_{how}$ times:
    \begin{equation}
        V^{(m+1)}(a, z) = u(i(a, z) - \sigma(a, z)) + \beta [\psi V^{(m)}(\sigma(a, z), z) + (1-\psi) \bar{V}^{(m)}(\sigma(a, z))]
    \end{equation}
\end{enumerate}
This significantly speeds up convergence as the Value Step is a linear operation\footnote{In principle, the value of a fixed policy $\sigma$ can be solved in closed form as $V_\sigma = (I - \beta P_\sigma)^{-1} U_\sigma$, where $P_\sigma$ is the transition matrix of the state space. However, for a grid size of $N_a N_z \approx 36,000$, the memory and computational overhead of inverting a $P_\sigma$ matrix with $1.3 \times 10^9$ elements makes the iterative successive approximation approach more efficient in terms of CPU cache and memory.}.

\subsection{\color{blue} Possible Improvements: Continuous Choice via Interpolation}
{\color{blue}
A significant enhancement to the discrete grid search is to allow for a continuous asset choice $a' \in [a_{min}, i]$. This is implemented by replacing the discrete maximization with a continuous optimizer and interpolating the continuation value:
\begin{equation}
    a^*(a, z) = \arg\max_{a' \in [a_{min}, i]} \left\{ u(i - a') + \beta \mathcal{I}(\text{Cont}, a', z) \right\}
\end{equation}
where $\mathcal{I}$ is an interpolating function (e.g., linear or cubic spline). Mathematically, this smooths the aggregate supply and demand functions $K_d(r)$ and $L_d(w)$, eliminating the "step" artifacts inherent in discrete grids and improving the convergence of the bisection algorithms in the general equilibrium solver.
}

\subsection{Coupled VFI for Distortions}
In the pre-reform state, the economy features idiosyncratic distortions $\tau \in \{\tau^+, \tau^-\}$. The solver maintains two coupled value functions $V^+(a, z)$ and $V^-(a, z)$. The expectation $\bar{V}(a')$ now accounts for the probability $p(z) = P(\tau = \tau^+ | z)$:
\begin{equation}
    \bar{V}(a') = \sum_{j} \pi_j \left[ p(z_j) V^+(a', z_j) + (1-p(z_j)) V^-(a', z_j) \right]
\end{equation}
The decision for an agent in state $(a, z, \tau)$ depends on the specific income $i(a, z, \tau)$, but the continuation value uses the shared $\bar{V}(a')$ since $\tau$ is redrawn whenever $z$ resets.

\section{Distributional and Transition Dynamics}

\subsection{Monte Carlo Binned Distribution}
The distribution $\mu_t(a, z)$ is tracked by simulating $N_{sim}$ agents. At any time $t$, agent $j$ has assets $a_j$. The mass is mapped to the asset grid via linear interpolation weights:
\begin{equation}
    \omega_{k+1} = \frac{a_j - a_k}{a_{k+1} - a_k}, \quad \omega_k = 1 - \omega_{k+1}
\end{equation}
This allows for calculating aggregate labor demand $L_{d,t}$ and capital demand $K_{d,t}$ as smooth functions of prices:
\begin{equation}
    K_{d,t}(w, r) = \sum_{i, j} \mu_t(a_i, z_j) k^*(a_i, z_j, w, r) \bm{1}_{\{\pi > w\}}
\end{equation}

\section{Algorithm B.2: Nested Price Path Iteration}
The general equilibrium transition path is computed via Algorithm B.2 (Appendix B.2), which decomposes the market clearing of capital and labor into nested price-path sequences.

\subsection{Price Path Sequences}
Let $\bm{w} = \{w_t\}_{t=0}^T$ and $\bm{r} = \{r_t\}_{t=0}^T$ be the sequences of wages and interest rates. The algorithm solves for fixed points of these sequences such that markets clear at every $t$.

\subsection{Nested Iteration Structure}
The implementation uses a nested loop structure:
\begin{enumerate}
    \item \textbf{Outer Loop (Interest Rates)}: Updates the guess for the interest rate path $\bm{r}^{(n)}$.
    \begin{enumerate}
        \item \textbf{Inner Loop (Wages)}: Given $\bm{r}^{(n)}$, iteratively updates the wage path $\bm{w}^{(k)}$.
        \begin{enumerate}
            \item \textbf{Backward Induction}: Solve for policies $a'_t(a, z)$ given $(w_t^{(k)}, r_t^{(n)})$ for $t = T-1, \dots, 0$.
            \item \textbf{Forward Simulation}: Simulate $N_{sim}$ agents forward from $\mu_0$ using $a'_t$ to obtain the sequence of binned distributions $\{\mu_t\}_{t=0}^T$.
            \item \textbf{Labor Clearing}: For each $t$, find the wage $w_t^*$ that clears the labor market \textit{given the fixed distribution} $\mu_t$:
            \begin{equation}
                L_{excess}(w_t^* | \mu_t, r_t^{(n)}) = 0
            \end{equation}
            \item \textbf{Wage Relaxation}: Update the wage path using damping $\eta_w$:
            \begin{equation}
                w_t^{(k+1)} = \eta_w w_t^* + (1-\eta_w) w_t^{(k)}
            \end{equation}
        \end{enumerate}
        \item \textbf{Capital Clearing}: For each $t$, find the interest rate $r_t^*$ that clears the capital market given the converged wage and distribution:
        \begin{equation}
            K_{excess}(r_t^* | \mu_t, w_t^{(converged)}) = 0
        \end{equation}
        \item \textbf{Interest Rate Relaxation}: Update the path using damping $\eta_r$:
        \begin{equation}
            r_t^{(n+1)} = \eta_r r_t^* + (1-\eta_r) r_t^{(n)}
        \end{equation}
    \end{enumerate}
\end{enumerate}

\subsection{Numerical Market Clearing (Bisections)}
The clearing prices $w_t^*$ and $r_t^*$ are found using a robust root-finding method:
\begin{itemize}
    \item \textbf{Bracketing Scan}: Before bisecting, the price space $[P_{min}, P_{max}]$ is scanned at $N_{scan}$ points to find a bracket $[a, b]$ where the excess demand function changes sign.
    \item \textbf{Bisection}: Once a bracket is found, a bisection method converges to the specific clearing price with a tolerance of $10^{-10}$. 
\end{itemize}

\subsection{Convergence Criteria}
The algorithm requires two conditions to stop:
\begin{enumerate}
    \item \textbf{Sequence Convergence}: The maximum change in price paths between iterations must be small: $\max_t |r_t^{(n+1)} - r_t^{(n)}| < \epsilon_{seq}$.
    \item \textbf{Market Clearing Gaps}: The actual excess demand residuals (calculated by re-evaluating the economy with the final prices) must satisfy $|ED_L| < \epsilon_{ED}$ and $|ED_K| < \epsilon_{ED}$.
\end{enumerate}
In the implementation, the default tolerances are $\epsilon_{seq} = 2 \times 10^{-4}$ for sequence convergence and $\epsilon_{ED} = 2 \times 10^{-3}$ for market clearing gaps. The relaxation parameters are $\eta_w = 0.35$ for wages and $\eta_r = 0.20$ for interest rates.

\section{Aggregate Variables and TFP}
Once the equilibrium prices $(w^*, r^*)$ and stationary distribution $\mu^*$ are computed, aggregate quantities are calculated as:
\begin{align}
    K &= \sum_{i,j} \mu(a_i, z_j) \cdot k^*(a_i, z_j) \cdot \bm{1}_{\{\pi > w\}} \\
    L_d &= \sum_{i,j} \mu(a_i, z_j) \cdot l^*(a_i, z_j) \cdot \bm{1}_{\{\pi > w\}} \\
    Y &= \sum_{i,j} \mu(a_i, z_j) \cdot y(a_i, z_j, k^*, l^*) \cdot \bm{1}_{\{\pi > w\}} \\
    s_e &= \sum_{i,j} \mu(a_i, z_j) \cdot \bm{1}_{\{\pi > w\}}
\end{align}
where $s_e$ is the share of entrepreneurs and $L_s = 1 - s_e$ is the labor supply. Total Factor Productivity (TFP) is then computed as:
\begin{equation}
    \text{TFP} = \frac{Y}{\left( K^\alpha L_s^{1-\alpha} \right)^{1-\nu}}
\end{equation}
This measure captures the efficiency losses from misallocation due to financial frictions.

\section{Calibration Parameters}
The model is calibrated following Buera \& Shin (2013), Table 1:
\begin{center}
\begin{tabular}{lcc}
\hline
Parameter & Symbol & Value \\
\hline
Risk aversion & $\sigma$ & 1.5 \\
Discount factor & $\beta$ & 0.904 \\
Capital share & $\alpha$ & 0.33 \\
Span of control & $1-\nu$ & 0.79 \\
Depreciation & $\delta$ & 0.06 \\
Pareto tail & $\eta$ & 4.15 \\
Ability persistence & $\psi$ & 0.894 \\
Collateral constraint & $\lambda$ & 1.35 \\
\hline
\multicolumn{3}{l}{\textit{Pre-reform distortions:}} \\
Tax wedge & $\tau^+$ & 0.57 \\
Subsidy wedge & $\tau^-$ & $-0.15$ \\
Correlation parameter & $q$ & 1.55 \\
\hline
\end{tabular}
\end{center}

\end{document}
