\documentclass{article}
\usepackage[utf8]{inputenc}
\usepackage{amsmath}
\usepackage{amssymb}
\usepackage{geometry}
\usepackage{bm}
\usepackage{xcolor}
\usepackage{hyperref}
\geometry{a4paper, margin=1in}

\title{Mathematical Foundations of the Buera \& Shin (2013) \\ Spectral Time Iteration Implementation}
\author{Piero De Dominicis \\ Bocconi University}
\date{\today}

\begin{document}

\maketitle

\section{The Economic Environment}
The economy is populated by a continuum of agents who choose to be either workers or entrepreneurs. Agents are heterogeneous in their asset holdings $a$ and entrepreneurial ability $z$.

\subsection{Preferences and Shocks}
Agents maximize expected lifetime utility:
\begin{equation}
    \mathbb{E}_0 \sum_{t=0}^{\infty} \beta^t \frac{c_t^{1-\sigma}-1}{1-\sigma}
\end{equation}
Entrepreneurial ability $z$ follows a reset process: with probability $\psi$, $z_{t+1} = z_t$; with probability $1-\psi$, a new ability is drawn $z' \sim \pi(z)$ from a Pareto distribution with tail parameter $\eta$. In the pre-reform state, agents also face idiosyncratic output wedges $\tau \in \{\tau^+, \tau^-\}$, where $\mathbb{P}(\tau = \tau^+ | z) = 1 - e^{-q z}$.

\section{Production and Optimal Firm Decisions}
An entrepreneur with ability $z$ and assets $a$ operates the technology:
\begin{equation} 
    y = z (k^\alpha l^{1-\alpha})^{1-\nu}
\end{equation}
where $\nu$ represents the span-of-control parameter (entrepreneur's share). The entrepreneur faces a collateral constraint $k \leq \lambda a$.

\subsection{Static Profit Maximization}
The entrepreneur's problem is solved in two stages. First, given capital $k$, the optimal labor $l^*(k, z)$ is found by solving:
\begin{equation}
    \max_l \{ z (k^\alpha l^{1-\alpha})^{1-\nu} - w l - (r+\delta) k \}
\end{equation}
The first-order condition with respect to $l$ is:
\begin{equation}
    (1-\nu)(1-\alpha) z k^{\alpha(1-\nu)} l^{(1-\alpha)(1-\nu)-1} = w
\end{equation}
Solving for $l$ yields the conditional labor demand:
\begin{equation}
    l^*(k, z) = \left[ \frac{(1-\nu)(1-\alpha) z}{w} \right]^{\frac{1}{1-(1-\alpha)(1-\nu)}} k^{\frac{\alpha(1-\nu)}{1-(1-\alpha)(1-\nu)}}
\end{equation}
Substituting $l^*(k, z)$ back into the profit function gives a profit function $\pi(k)$ that is concave in $k$. The unconstrained capital demand $k^{unc}$ satisfies $\pi'(k^{unc}) = r+\delta$. If $k^{unc} > \lambda a$, the entrepreneur is borrowing-constrained and sets $k^* = \lambda a$.

\section{The Recursive Household Problem}
The agent's state is $(a, z)$. The total income is $y(a, z) = \max\{\pi(a, z, k^*), w\} + (1+r)a$. The Bellman equation is:
\begin{equation}
    V(a, z) = \max_{a'} \left\{ \frac{(y(a, z) - a')^{1-\sigma}-1}{1-\sigma} + \beta \text{Cont}(a', z) \right\}
\end{equation}
where the continuation value is:
\begin{equation}
    \text{Cont}(a', z) = \psi V(a', z) + (1-\psi) \int V(a', z') \pi(z') dz'
\end{equation}

\subsection{Derivation of the Euler Equation}
To solve the household's problem, we substitute $c_t = y(a_t, z_t) - a_{t+1}$ into the value function:
\begin{equation}
    V(a_t, z_t) = \max_{a_{t+1}} \left\{ \frac{(y(a_t, z_t) - a_{t+1})^{1-\sigma} - 1}{1-\sigma} + \beta \mathbb{E}[V(a_{t+1}, z_{t+1}) | z_t] \right\}
\end{equation}
The first-order condition with respect to $a_{t+1}$ is:
\begin{equation}
    -(y(a_t, z_t) - a_{t+1})^{-\sigma} + \beta \mathbb{E}\left[ \frac{\partial V(a_{t+1}, z_{t+1})}{\partial a_{t+1}} \right] = 0
\end{equation}
From the Envelope Theorem, the derivative of the value function with respect to assets is:
\begin{equation}
    \frac{\partial V(a_t, z_t)}{\partial a_t} = (y(a_t, z_t) - a_{t+1})^{-\sigma} \cdot \frac{\partial y(a_t, z_t)}{\partial a_t} = (1+r) c_t^{-\sigma}
\end{equation}
Leading to the stochastic Euler equation: $c_t^{-\sigma} = \beta (1+r) \mathbb{E}_t [ c_{t+1}^{-\sigma} ]$.

\section{Spectral Discretization and Setup}
The policy function $a'(a, z)$ is approximated by a bivariate Chebyshev polynomial $\hat{a}(a, z; \bm{\gamma})$.

\subsection{Chebyshev Collocation Nodes}
The state space is discretized using the zeros of the $N$-th degree Chebyshev polynomial $\xi \in [-1, 1]$:
\begin{equation}
    \xi_k = \cos \left( \frac{2k - 1}{2N} \pi \right), \quad k \in \{1, \dots, N\}
\end{equation}
The nodes are mapped to the physical state space $(a, z)$. For assets, we use a log-linear mapping to cluster nodes near the borrowing constraint:
\begin{equation}
    a_k = \exp\left( \frac{\ln(a_{max} + \text{shift}) + \ln(a_{min} + \text{shift})}{2} + \xi_k \frac{\ln(a_{max} + \text{shift}) - \ln(a_{min} + \text{shift})}{2} \right) - \text{shift}
\end{equation}
while ability nodes $z_j$ are mapped linearly.

\section{Spectral Time Iteration Logic}
\subsection{Pointwise Euler Inversion}
To solve for the optimal policy, we utilize a recursive operator that linearizes the Global Euler Equation. At each collocation node $(a_i, z_j)$, we define the \textbf{Target Function} $F(a')$ as the difference between current marginal utility and expected future marginal utility:
\begin{equation}
    F(a'; a_i, z_j, \bm{\gamma}^{(n)}) \equiv \underbrace{(y_i - a')^{-\sigma}}_{\text{Current MU}} - \beta(1+r) \underbrace{\mathcal{E}(a'; \bm{\gamma}^{(n)})}_{\text{Continuation MU}}
\end{equation}
where $\mathcal{E}$ is the expectation over future states $(\hat{a}_i, z')$ given the \textbf{known} continuation policy $\bm{\gamma}^{(n)}$:
\begin{equation}
    \mathcal{E}(a'; \bm{\gamma}^{(n)}) = \psi \left( y(a', z_j) - \bm{\Phi}(a', z_j)^\top \bm{\gamma}^{(n)} \right)^{-\sigma} + (1-\psi) \sum_k \pi_k \left( y(a', z_k) - \bm{\Phi}(a', z_k)^\top \bm{\gamma}^{(n)} \right)^{-\sigma}
\end{equation}
Since $\mathcal{E}$ is fixed for a given guess of $a'$, we can identify the optimal $a^*_{target}$ by direct inversion:
\begin{equation}
    a^*_{target, i} = \max\{a_{min}, y_i - \left[ \beta(1+r) \mathcal{E}(a'; \bm{\gamma}^{(n)}) \right]^{-1/\sigma} \}
\end{equation}
This formulation depends exclusively on the grid points $(a_i, z_j)$ and the candidate savings $a'$, with the future behavior already "baked" into the coefficients $\bm{\gamma}^{(n)}$.

\subsection{Damped Coefficient Update}
The algorithm computes the new policy nodes $\bm{A}^*_{new}$ by applying a damping parameter $\theta \in (0, 1]$ to ensure numerical stability:
\begin{equation}
    \bm{A}^{*}_{new} = (1-\theta) \bm{A}^{*}_{old} + \theta \bm{A}^*_{target}
\end{equation}
The updated spectral coefficients $\bm{\gamma}^{(n+1)}$ are then obtained by solving the linear basis transformation:
\begin{equation}
    \bm{\gamma}^{(n+1)} = \bm{T}^{-1} \bm{A}^{*}_{new}
\end{equation}
where $\bm{T}$ is the matrix of Chebyshev basis functions evaluated at the collocation nodes.




\section{Distributional and Transition Dynamics}

\subsection{Transition Matrix Construction}
The distribution is tracked on a dense histogram grid $(a^h_i, z^h_j)$ with $N_a^h \times N_z^h$ points. Given the continuous spectral policy $\hat{a}'(a, z; \bm{\gamma})$, we construct a sparse Markov transition matrix $Q$ as follows.

For each state $(a^h_i, z^h_j)$, we compute the optimal savings $a' = \hat{a}'(a^h_i, z^h_j; \bm{\gamma})$. Since $a'$ generally falls between grid points, we use linear interpolation weights:
\begin{equation}
    a' \in [a^h_k, a^h_{k+1}] \implies \omega_{k+1} = \frac{a' - a^h_k}{a^h_{k+1} - a^h_k}, \quad \omega_k = 1 - \omega_{k+1}
\end{equation}
The transition probability from state $(a^h_i, z^h_j)$ to $(a^h_m, z^h_n)$ is:
\begin{equation}
    Q_{(i,j) \to (m,n)} = \omega_m \cdot \mathbb{P}(z^h_n | z^h_j)
\end{equation}
where $\mathbb{P}(z' | z) = \psi \cdot \bm{1}_{z'=z} + (1-\psi) \cdot \pi(z')$.

\subsection{Stationary Distribution}
The stationary distribution $\mu^*$ satisfies $\mu^* = Q^\top \mu^*$ and is computed via power iteration:
\begin{equation}
    \mu^{(n+1)} = \frac{Q^\top \mu^{(n)}}{\| Q^\top \mu^{(n)} \|_1}
\end{equation}
Convergence is achieved when $\| \mu^{(n+1)} - \mu^{(n)} \|_\infty < 10^{-10}$.

\subsection{Forward Simulation for Transition}
During the transition path, the distribution $\mu_t$ evolves according to:
\begin{equation}
    \mu_{t+1}(a', z') = \sum_{a, z} \mathbb{P}(a' | a, z; \bm{\gamma}_t) \mathbb{P}(z' | z) \mu_t(a, z)
\end{equation}
where $\bm{\gamma}_t$ are the time-varying spectral coefficients computed via backward induction.

\section{Coupled Value Functions for Pre-Reform Distortions}
In the pre-reform economy with idiosyncratic distortions $\tau \in \{\tau^+, \tau^-\}$, we solve for two coupled policy functions $\hat{a}^+(a, z; \bm{\gamma}^+)$ and $\hat{a}^-(a, z; \bm{\gamma}^-)$. The expectation operator becomes:
\begin{align}
    \mathcal{E}^+(a'; \bm{\gamma}^+, \bm{\gamma}^-) &= \psi \cdot \mu^+(a', z) + (1-\psi) \sum_k \pi_k \left[ p(z_k) \mu^+(a', z_k) + (1-p(z_k)) \mu^-(a', z_k) \right] \\
    \mathcal{E}^-(a'; \bm{\gamma}^+, \bm{\gamma}^-) &= \psi \cdot \mu^-(a', z) + (1-\psi) \sum_k \pi_k \left[ p(z_k) \mu^+(a', z_k) + (1-p(z_k)) \mu^-(a', z_k) \right]
\end{align}
where $\mu^\pm(a, z) = (y^\pm(a, z) - \hat{a}^\pm(a, z))^{-\sigma}$ is the marginal utility under each distortion state, and $p(z) = 1 - e^{-qz}$ is the probability of $\tau = \tau^+$ conditional on ability $z$.

\section{General Equilibrium: Nested Price Clearing}
The general equilibrium prices $(w^*, r^*)$ are found using Algorithm B.2 from Buera \& Shin (2013):

\subsection{Nested Loop Structure}
\begin{enumerate}
    \item \textbf{Outer Loop (Interest Rate)}: Bisection on $r \in [r_{min}, r_{max}]$ to clear the capital market.
    \begin{enumerate}
        \item \textbf{Inner Loop (Wage)}: Given $r^{(n)}$, bisection on $w$ to clear the labor market:
        \begin{itemize}
            \item Solve the spectral policy $\bm{\gamma}(w, r^{(n)})$
            \item Compute stationary distribution $\mu(w, r^{(n)})$
            \item Evaluate excess labor demand: $ED_L(w) = L_d(\mu) - (1 - s_e(\mu))$
        \end{itemize}
        \item With converged $w^*(r^{(n)})$, evaluate capital market excess demand:
        \begin{equation}
            ED_K(r) = K_d(\mu) - A_s(\mu)
        \end{equation}
    \end{enumerate}
\end{enumerate}

\subsection{Smoothness of Spectral Methods}
The key advantage of spectral approximation is that the policy function $\hat{a}'(a, z)$ is \textit{globally smooth} (infinitely differentiable). This ensures that aggregate demand functions $K_d(w, r)$ and $L_d(w, r)$ are continuous and differentiable, eliminating the ``step'' artifacts that arise from discrete grid methods and improving convergence of the bisection algorithm.

\section{Calibration Parameters}
\begin{center}
\begin{tabular}{lcc}
\hline
Parameter & Symbol & Value \\
\hline
Risk aversion & $\sigma$ & 1.5 \\
Discount factor & $\beta$ & 0.904 \\
Capital share & $\alpha$ & 0.33 \\
Span of control & $1-\nu$ & 0.79 \\
Depreciation & $\delta$ & 0.06 \\
Pareto tail & $\eta$ & 4.15 \\
Ability persistence & $\psi$ & 0.894 \\
Collateral constraint & $\lambda$ & 1.35 \\
\hline
\multicolumn{3}{l}{\textit{Spectral parameters:}} \\
Chebyshev nodes (assets) & $N_a^{cheb}$ & 15 \\
Chebyshev nodes (ability) & $N_z^{cheb}$ & 15 \\
Histogram grid (assets) & $N_a^h$ & 600 \\
Histogram grid (ability) & $N_z^h$ & 40 \\
\hline
\end{tabular}
\end{center}

\end{document}
