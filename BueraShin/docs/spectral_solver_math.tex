\documentclass{article}
\usepackage[utf8]{inputenc}
\usepackage[T1]{fontenc}
\usepackage{amsmath, amssymb}
\usepackage{hyperref}

\title{Mathematical Appendix: Spectral Policy Solver}
\author{Documentation for Buera \& Shin (2010) v2 Implementation}
\date{\today}

\begin{document}

\maketitle

\section{The Recursive Problem}
The agent faces an infinite-horizon consumer-entrepreneur problem. The Bellman equation is given by:
\begin{equation}
V(a, z) = \max_{c, a'} \frac{c^{1-\sigma}}{1-\sigma} + \beta \mathbb{E}_{z'} [V(a', z')]
\end{equation}
Subject to the budget constraint:
\begin{equation}
c + a' = \max\{\pi(a, z), w\} + (1+r)a
\end{equation}
where $a$ is assets, $z$ is entrepreneurial productivity, $\pi(a, z)$ is the profit from entrepreneurship, $w$ is the equilibrium wage, and $r$ is the interest rate.

\section{The Optimality Condition}
The First-Order Condition with respect to $a'$ (the Euler Equation) is:
\begin{equation}
u'(c) = \beta (1+r) \mathbb{E}_{z'} [u'(c')]
\end{equation}
Using the CRRA utility $u(c) = \frac{c^{1-\sigma}}{1-\sigma}$, we have $u'(c) = c^{-\sigma}$. Solving for current consumption $c$:
\begin{equation}
c = \left( \beta (1+r) \mathbb{E}_{z'} [ (c')^{-\sigma} ] \right)^{-1/\sigma}
\end{equation}

\section{Spectral Approximation}
The policy function $a' = f(a, z)$ is approximated using a bivariate Chebyshev expansion:
\begin{equation}
\hat{f}(a, z; \Theta) = \sum_{i=0}^{N_a-1} \sum_{j=0}^{N_z-1} \theta_{i,j} T_i(\phi_a(a)) T_j(\phi_z(z))
\end{equation}
where:
\begin{itemize}
    \item $\Theta = \{\theta_{i,j}\}$ is the matrix of \texttt{coeffs}.
    \item $T_i(x)$ are Chebyshev polynomials of the first kind.
    \item $\phi_a(a)$ and $\phi_z(z)$ are linear mappings from the physical domain to $[-1, 1]$.
\end{itemize}

\section{Implementation: Fixed-Point Update}
The routine \texttt{solve\_policy\_bivariate\_update} performs the following steps:
\begin{enumerate}
    \item \textbf{Collocation}: Defines the problem at the $N_a \times N_z$ Chebyshev nodes $(a_n, z_m)$.
    \item \textbf{Quadrature}: The expectation $\mathbb{E}_{z'} [ u'(c') ]$ is computed as:
    \begin{equation}
    \mathbb{E}[MU'] = \psi \cdot u'(c'(a', z_m)) + (1-\psi) \int_{z'} u'(c'(a', z')) dG(z')
    \end{equation}
    The integral is discretized using a fine Pareto quadrature grid.
    \item \textbf{Iterative Step}: Given the current coefficients $\Theta^{(k)}$, a target value $f_{target}$ is calculated for each node.
    \item \textbf{Projection}: The updated coefficients are found by projecting the targets back onto the Chebyshev basis:
    \begin{equation}
    \Theta^{(k+1)} = \mathbf{T}^{-1} \cdot \mathbf{f}_{target}
    \end{equation}
\end{enumerate}

\end{document}
