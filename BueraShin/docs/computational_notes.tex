\documentclass[11pt,a4paper]{article}

% Packages
\usepackage[utf8]{inputenc}
\usepackage[T1]{fontenc}
\usepackage{amsmath,amssymb,amsthm}
\usepackage{mathtools}
\usepackage{geometry}
\usepackage{booktabs}
\usepackage{fancyvrb}
\usepackage{framed}
\usepackage{graphicx}
\usepackage{hyperref}
\usepackage[numbers]{natbib}
\usepackage{setspace}

\geometry{margin=1in}
\onehalfspacing

% Custom commands
\newcommand{\E}{\mathbb{E}}
\newcommand{\R}{\mathbb{R}}
\newcommand{\N}{\mathbb{N}}
\newcommand{\1}{\mathbf{1}}
\DeclareMathOperator*{\argmax}{arg\,max}

% Title
\title{Computational Methods for Buera \& Shin (2013):\\
\large Financial Frictions and the Persistence of History}
\author{Replication Notes}
\date{January 2025}

\begin{document}

\maketitle

\begin{abstract}
This document provides detailed computational notes for replicating the stationary equilibrium and Figure 2 of \cite{buera2013financial}. We describe the model setup, the individual optimization problem, value function iteration, computation of the stationary distribution, and the general equilibrium algorithm. Implementation details for Python with Numba acceleration are provided.
\end{abstract}

\tableofcontents
\newpage

%===============================================================================
\section{Model Environment}
%===============================================================================

\subsection{Preferences}

There is a continuum of infinitely-lived agents with measure one. Agents have CRRA preferences over consumption:
\begin{equation}
U = \E_0 \sum_{t=0}^{\infty} \beta^t u(c_t), \quad u(c) = \frac{c^{1-\sigma} - 1}{1-\sigma}
\end{equation}
where $\beta \in (0,1)$ is the discount factor and $\sigma > 0$ is the coefficient of relative risk aversion. For $\sigma = 1$, we have $u(c) = \log(c)$.

\subsection{Technology}

Each agent is endowed with entrepreneurial ability $e \geq 1$. An entrepreneur with ability $e$ operates a production technology:
\begin{equation}
y = f(e, k, \ell) = e \cdot (k^\alpha \ell^{1-\alpha})^{1-\nu}
\end{equation}
where:
\begin{itemize}
    \item $k$ is capital input
    \item $\ell$ is labor input
    \item $\alpha \in (0,1)$ is the capital share in variable factors
    \item $\nu \in (0,1)$ is the entrepreneur's share (span-of-control parameter)
\end{itemize}

The span-of-control parameter $\nu$ implies decreasing returns to scale in $(k, \ell)$, which bounds firm size even with perfect credit markets.

\subsection{Entrepreneurial Ability Process}

Ability $e$ follows a Markov process. With probability $\psi$, ability remains unchanged. With probability $1-\psi$, a new ability is drawn from the stationary Pareto distribution:
\begin{equation}
\Pr(e' | e) = \psi \cdot \1_{e' = e} + (1-\psi) \cdot g(e')
\end{equation}
where the stationary distribution has Pareto density:
\begin{equation}
g(e) = \eta \cdot e^{-(\eta+1)}, \quad e \geq 1
\end{equation}
with CDF $G(e) = 1 - e^{-\eta}$ and tail parameter $\eta > 0$.

\subsection{Financial Frictions}

Agents face a collateral constraint on capital:
\begin{equation}
k \leq \lambda \cdot a
\end{equation}
where $a$ is the agent's asset holdings and $\lambda \geq 1$ is the financial friction parameter:
\begin{itemize}
    \item $\lambda = 1$: Financial autarky (no external borrowing)
    \item $\lambda = \infty$: Perfect credit markets
    \item $\lambda \in (1, \infty)$: Intermediate financial development
\end{itemize}

\subsection{Markets}

Markets are competitive:
\begin{itemize}
    \item Labor market: wage $w$
    \item Capital rental market: rental rate $r + \delta$ where $r$ is the interest rate and $\delta$ is depreciation
    \item Asset market: agents can save at rate $r$ (no borrowing beyond collateral)
\end{itemize}

%===============================================================================
\section{Individual Optimization Problem}
%===============================================================================

\subsection{Occupational Choice}

Each period, an agent with state $(a, e)$ chooses to be either:
\begin{enumerate}
    \item \textbf{Worker}: Supplies 1 unit of labor, earns wage $w$
    \item \textbf{Entrepreneur}: Operates a firm, earns profit $\pi(a, e; w, r, \lambda)$
\end{enumerate}

The agent chooses the occupation that maximizes income:
\begin{equation}
\text{Occupation} = \begin{cases}
\text{Entrepreneur} & \text{if } \pi(a, e; w, r, \lambda) > w \\
\text{Worker} & \text{otherwise}
\end{cases}
\end{equation}

\subsection{Entrepreneur's Static Problem}

An entrepreneur with assets $a$ and ability $e$ solves:
\begin{equation}
\pi(a, e; w, r, \lambda) = \max_{k, \ell \geq 0} \left\{ e \cdot (k^\alpha \ell^{1-\alpha})^{1-\nu} - w\ell - (r+\delta)k \right\}
\end{equation}
subject to:
\begin{equation}
k \leq \lambda \cdot a
\end{equation}

\subsubsection{First-Order Conditions}

The FOCs for an interior solution are:
\begin{align}
\frac{\partial f}{\partial k} &= (r + \delta) \quad \Rightarrow \quad \alpha(1-\nu) \cdot e \cdot (k^\alpha \ell^{1-\alpha})^{-\nu} \cdot k^{\alpha-1} \ell^{1-\alpha} = r + \delta \\
\frac{\partial f}{\partial \ell} &= w \quad \Rightarrow \quad (1-\alpha)(1-\nu) \cdot e \cdot (k^\alpha \ell^{1-\alpha})^{-\nu} \cdot k^{\alpha} \ell^{-\alpha} = w
\end{align}

\subsubsection{Unconstrained Optimal Capital}

Combining the FOCs, the unconstrained optimal capital $k^*$ satisfies:
\begin{equation}
k^* = \left[ \frac{\alpha(1-\nu)}{r+\delta} \right]^{\frac{1-(1-\alpha)(1-\nu)}{\nu}} \cdot \left[ \frac{(1-\alpha)(1-\nu)}{w} \right]^{\frac{(1-\alpha)(1-\nu)}{\nu}} \cdot e^{1/\nu}
\end{equation}

\textbf{Derivation}: Define the span $s \equiv 1 - \nu$. From the FOCs:
\begin{align}
\frac{k}{\ell} &= \frac{\alpha}{1-\alpha} \cdot \frac{w}{r+\delta} \\
\ell &= \left[ \frac{(1-\alpha)s \cdot e \cdot k^{\alpha s}}{w} \right]^{\frac{1}{1-(1-\alpha)s}}
\end{align}

Substituting into the capital FOC and solving yields the expression above.

\subsubsection{Constrained Solution}

The optimal capital is:
\begin{equation}
\hat{k}(a, e) = \min\{k^*, \lambda a\}
\end{equation}

Given $\hat{k}$, the optimal labor is:
\begin{equation}
\hat{\ell}(a, e) = \left[ \frac{(1-\alpha)(1-\nu) \cdot e \cdot \hat{k}^{\alpha(1-\nu)}}{w} \right]^{\frac{1}{1-(1-\alpha)(1-\nu)}}
\end{equation}

And profit is:
\begin{equation}
\pi(a, e) = e \cdot (\hat{k}^\alpha \hat{\ell}^{1-\alpha})^{1-\nu} - w\hat{\ell} - (r+\delta)\hat{k}
\end{equation}

\subsection{Dynamic Problem: Bellman Equation}

The agent's dynamic optimization problem is:
\begin{equation}
V(a, e) = \max_{a' \geq 0} \left\{ u(c) + \beta \E[V(a', e') | e] \right\}
\end{equation}
subject to the budget constraint:
\begin{equation}
c + a' = y(a, e) + (1+r)a
\end{equation}
where:
\begin{equation}
y(a, e) = \max\{w, \pi(a, e)\}
\end{equation}

\subsubsection{Expected Continuation Value}

Given the ability transition process:
\begin{equation}
\E[V(a', e') | e] = \psi \cdot V(a', e) + (1-\psi) \sum_{e'} g(e') V(a', e')
\end{equation}

%===============================================================================
\section{Value Function Iteration}
%===============================================================================

\subsection{Discretization}

\subsubsection{Ability Grid}

Following the paper (page 235), we discretize ability into $n_e = 40$ grid points $\{e_1, \ldots, e_{40}\}$:

\begin{enumerate}
    \item Points $j = 1, \ldots, 38$: Equidistant in CDF space
    \begin{equation}
    G(e_j) = 0.633 + \frac{j-1}{37}(0.998 - 0.633), \quad j = 1, \ldots, 38
    \end{equation}

    \item Points $j = 39, 40$: Capture the tail
    \begin{equation}
    G(e_{39}) = 0.999, \quad G(e_{40}) = 0.9995
    \end{equation}
\end{enumerate}

Grid points are computed via the inverse CDF:
\begin{equation}
e_j = (1 - G(e_j))^{-1/\eta}
\end{equation}

Probability masses:
\begin{equation}
\text{Pr}(e_1) = \frac{G(e_1)}{G(e_{40})}, \quad \text{Pr}(e_j) = \frac{G(e_j) - G(e_{j-1})}{G(e_{40})}, \; j = 2, \ldots, 40
\end{equation}

\subsubsection{Asset Grid}

We use $n_a$ grid points with curvature scaling to concentrate points near zero:
\begin{equation}
a_i = a_{\min} + (a_{\max} - a_{\min}) \left( \frac{i-1}{n_a - 1} \right)^\gamma, \quad i = 1, \ldots, n_a
\end{equation}
where $\gamma = 2$ provides quadratic spacing. Typical values: $n_a = 501$, $a_{\min} = 10^{-6}$, $a_{\max} = 4000$.

\subsection{VFI Algorithm}

\begin{framed}
\noindent\textbf{Algorithm 1: Value Function Iteration}
\begin{Verbatim}[fontsize=\small]
INPUT: Grids (a_i), (e_j), probabilities Pr(e_j), prices (w,r), parameters
INITIALIZE: V^0(a_i, e_j) = u(w + r*a_i)/(1-beta) for all i,j
SET: tolerance eps = 1e-5, iteration n = 0

REPEAT:
  n = n + 1
  FOR j = 1,...,n_e:  [Can parallelize over j]
    Compute expected value:
      EV_j(a') = psi*V^{n-1}(a',e_j) + (1-psi)*SUM_j' Pr(e_j')*V^{n-1}(a',e_j')
    FOR i = 1,...,n_a:
      Solve entrepreneur: (pi_ij, k_ij, l_ij) = SolveEntrepreneur(a_i, e_j, w, r, lambda)
      Occupation: y_ij = max{w, pi_ij}
      Income: I_ij = y_ij + (1+r)*a_i
      Optimal savings: a'_ij = argmax_{a' < I_ij} {u(I_ij - a') + beta*EV_j(a')}
      Update: V^n(a_i, e_j) = u(I_ij - a'_ij) + beta*EV_j(a'_ij)
  Compute: diff = max_{i,j} |V^n(a_i, e_j) - V^{n-1}(a_i, e_j)|
UNTIL: diff < eps

OUTPUT: Value function V, policy functions a'(a,e), occ(a,e)
\end{Verbatim}
\end{framed}

\subsection{Solving the Savings Problem}

For each state $(a_i, e_j)$ with income $I$, we find:
\begin{equation}
a'_* = \argmax_{k: a_k < I} \left\{ u(I - a_k) + \beta \cdot EV(a_k) \right\}
\end{equation}

\textbf{Grid search}: Since $u$ is strictly concave and the grid is finite, we can use:
\begin{enumerate}
    \item Linear search: Check all feasible $a_k$ (simple but $O(n_a)$)
    \item Binary search: Exploit concavity to find the peak ($O(\log n_a)$)
\end{enumerate}

In practice, linear search with early stopping (break when $c \leq 0$) works well since the asset grid is ordered.

%===============================================================================
\section{Stationary Distribution}
%===============================================================================

\subsection{Transition Matrix}

The state space is $\mathcal{S} = \{1, \ldots, n_a\} \times \{1, \ldots, n_e\}$ with $|\mathcal{S}| = n_a \times n_e$ states.

We index states as $s = (i-1) \cdot n_e + j$ for $(a_i, e_j)$.

The transition matrix $Q \in \R^{|\mathcal{S}| \times |\mathcal{S}|}$ has entries:
\begin{equation}
Q_{s', s} = \Pr(\text{state } s' | \text{state } s) = \Pr(a' = a_{i'}, e' = e_{j'} | a = a_i, e = e_j)
\end{equation}

Given the policy function $a'(a_i, e_j) = a_{i'(i,j)}$:
\begin{equation}
Q_{s', s} = \begin{cases}
\psi + (1-\psi) \cdot \text{Pr}(e_{j'}) & \text{if } i' = i'(i,j) \text{ and } j' = j \\
(1-\psi) \cdot \text{Pr}(e_{j'}) & \text{if } i' = i'(i,j) \text{ and } j' \neq j \\
0 & \text{otherwise}
\end{cases}
\end{equation}

\subsection{Computing the Stationary Distribution}

The stationary distribution $\mu^* \in \R^{|\mathcal{S}|}$ satisfies:
\begin{equation}
\mu^* = Q \mu^*, \quad \sum_s \mu^*_s = 1
\end{equation}

\subsubsection{Method 1: Power Iteration}

\begin{framed}
\noindent\textbf{Algorithm 2: Power Iteration for Stationary Distribution}
\begin{Verbatim}[fontsize=\small]
INITIALIZE: mu^0 = (1/|S|) * 1  (uniform distribution)

REPEAT:
  mu^{n+1} = Q * mu^n
  mu^{n+1} = mu^{n+1} / ||mu^{n+1}||_1  (normalize)
UNTIL: ||mu^{n+1} - mu^n||_inf < eps

OUTPUT: mu* = mu^{n+1}
\end{Verbatim}
\end{framed}

\subsubsection{Method 2: Eigenvalue Method}

Find the eigenvector corresponding to eigenvalue 1:
\begin{equation}
Q v = v \quad \Rightarrow \quad \mu^* = \frac{v}{\|v\|_1}
\end{equation}

In practice, use sparse matrix methods since $Q$ has at most $n_e$ non-zero entries per column.

\subsection{Sparse Matrix Implementation}

The transition matrix $Q$ is highly sparse. For each state $s = (i, j)$:
\begin{itemize}
    \item There is exactly one destination asset index $i' = i'(i,j)$
    \item But $n_e$ possible destination ability indices $j' \in \{1, \ldots, n_e\}$
\end{itemize}

Total non-zero entries: $n_a \times n_e \times n_e$.

Store in Compressed Sparse Row (CSR) format for efficient matrix-vector multiplication.

%===============================================================================
\section{Aggregate Variables}
%===============================================================================

Given the stationary distribution $\mu(a, e)$ (reshaped to $n_a \times n_e$), we compute:

\subsection{Aggregate Capital}
\begin{equation}
K = \sum_{i,j} \mu(a_i, e_j) \cdot \hat{k}(a_i, e_j) \cdot \1_{\text{entrepreneur}}(a_i, e_j)
\end{equation}

\subsection{Aggregate Labor Demand}
\begin{equation}
L^d = \sum_{i,j} \mu(a_i, e_j) \cdot \hat{\ell}(a_i, e_j) \cdot \1_{\text{entrepreneur}}(a_i, e_j)
\end{equation}

\subsection{Aggregate Output}
\begin{equation}
Y = \sum_{i,j} \mu(a_i, e_j) \cdot f(e_j, \hat{k}(a_i, e_j), \hat{\ell}(a_i, e_j)) \cdot \1_{\text{entrepreneur}}(a_i, e_j)
\end{equation}

\subsection{Aggregate Assets}
\begin{equation}
A = \sum_{i,j} \mu(a_i, e_j) \cdot a_i
\end{equation}

\subsection{Share of Entrepreneurs}
\begin{equation}
\text{share}_e = \sum_{i,j} \mu(a_i, e_j) \cdot \1_{\text{entrepreneur}}(a_i, e_j)
\end{equation}

\subsection{External Finance}
\begin{equation}
\text{ExtFin} = \sum_{i,j} \mu(a_i, e_j) \cdot \max\{0, \hat{k}(a_i, e_j) - a_i\} \cdot \1_{\text{entrepreneur}}(a_i, e_j)
\end{equation}

\subsection{Total Factor Productivity}

Using standard capital share of 1/3:
\begin{equation}
\text{TFP} = \frac{Y}{K^{1/3} L^{2/3}}
\end{equation}

%===============================================================================
\section{General Equilibrium}
%===============================================================================

\subsection{Market Clearing Conditions}

\subsubsection{Labor Market}
Labor supply equals labor demand:
\begin{equation}
L^s = 1 - \text{share}_e = L^d
\end{equation}
Workers supply 1 unit each, so total labor supply is the fraction of workers.

\subsubsection{Capital Market}
Capital demand equals total assets:
\begin{equation}
K = A
\end{equation}
In equilibrium, entrepreneurs' capital is funded by all agents' savings.

\subsection{Equilibrium Algorithm}

\begin{framed}
\noindent\textbf{Algorithm 3: General Equilibrium Computation}
\begin{Verbatim}[fontsize=\small]
INPUT: Initial guesses (w^0, r^0), tolerance tau, max iterations N
INITIALIZE: best_error = inf

FOR n = 1,...,N:
  Solve VFI with (w^{n-1}, r^{n-1}) -> policies (a'(.), occ(.))
  Compute stationary distribution mu
  Compute aggregates: K, L^d, A, share_e

  Excess demands:
    ExcL = L^d - (1 - share_e)
    ExcK = K - A

  IF |ExcL| < tau AND |ExcK| < tau:
    CONVERGED

  Update prices:
    w_new = w^{n-1} * (1 + gamma_w * ExcL)
    r_new = r^{n-1} + gamma_r * ExcK

  Damped update:
    w^n = theta * w^{n-1} + (1-theta) * w_new
    r^n = theta * r^{n-1} + (1-theta) * r_new

  Apply bounds: w^n in [0.01, 2.0], r^n in [-0.06, 0.12]

OUTPUT: Equilibrium (w*, r*) and aggregates
\end{Verbatim}
\end{framed}

\textbf{Tuning parameters}:
\begin{itemize}
    \item Step sizes: $\gamma_w = 0.3$, $\gamma_r = 0.01$
    \item Damping: $\theta = 0.5$
    \item Tolerance: $\tau = 10^{-3}$
\end{itemize}

\subsection{Intuition for Price Adjustments}

\begin{itemize}
    \item \textbf{Excess labor demand} ($L^d > L^s$): Raise wage $w$ to reduce labor demand and encourage more agents to become workers
    \item \textbf{Excess capital demand} ($K > A$): Raise interest rate $r$ to reduce capital demand (higher rental cost) and encourage more savings
\end{itemize}

%===============================================================================
\section{Implementation Details}
%===============================================================================

\subsection{Numba JIT Compilation}

Key functions are compiled with \texttt{@njit(cache=True)} for C-level performance:
\begin{itemize}
    \item \texttt{solve\_entrepreneur}: Static profit maximization
    \item \texttt{utility}: CRRA utility evaluation
    \item \texttt{bellman\_operator}: Main VFI iteration
    \item \texttt{compute\_aggregates\_on\_grid}: Aggregate computation
\end{itemize}

\subsection{Parallelization}

The Bellman operator is parallelized over ability states using \texttt{@njit(parallel=True)} with \texttt{prange}:
\begin{verbatim}
@njit(cache=True, parallel=True)
def bellman_operator(...):
    for i_z in prange(n_z):  # Parallel loop
        for i_a in range(n_a):
            # Solve individual problem
\end{verbatim}

This provides near-linear speedup with the number of CPU cores.

\subsection{Memory Efficiency}

\begin{itemize}
    \item Use sparse matrices for the transition matrix $Q$
    \item Pre-compute expected values $EV(a', e)$ before the inner loops
    \item Store policy functions as integer indices rather than values
\end{itemize}

\subsection{Numerical Stability}

\begin{itemize}
    \item Bound rental rate: $r + \delta > 10^{-8}$
    \item Bound wage: $w > 10^{-8}$
    \item Bound consumption: $c > 10^{-10}$ before computing utility
    \item Use $-10^{10}$ as utility for infeasible consumption
\end{itemize}

%===============================================================================
\section{Calibration}
%===============================================================================

\begin{table}[h]
\centering
\begin{tabular}{lcl}
\toprule
Parameter & Value & Target/Source \\
\midrule
$\sigma$ & 1.5 & Standard value \\
$\beta$ & 0.904 & Interest rate $\approx 4.5\%$ \\
$\alpha$ & 0.33 & Capital share \\
$\nu$ & 0.21 & Span-of-control \\
$\delta$ & 0.06 & Depreciation rate \\
$\eta$ & 4.15 & Top 10\% employment share \\
$\psi$ & 0.894 & Exit rate of establishments \\
\bottomrule
\end{tabular}
\caption{Baseline calibration from \cite{buera2013financial}}
\end{table}

%===============================================================================
\section{Computational Performance}
%===============================================================================

\begin{table}[h]
\centering
\begin{tabular}{lcc}
\toprule
Component & Time per $\lambda$ & Notes \\
\midrule
VFI (per iteration) & $\sim 2$s & $501 \times 40$ states \\
VFI convergence & $\sim 50-200$ iter & Depends on initial guess \\
Stationary distribution & $\sim 0.5$s & Power iteration \\
GE iteration & $\sim 10-50$ iter & Price adjustment \\
\midrule
Total per $\lambda$ & $\sim 20-100$s & Varies with convergence \\
\bottomrule
\end{tabular}
\caption{Approximate computation times (single core, 3.5GHz)}
\end{table}

%===============================================================================
\section{Results Summary}
%===============================================================================

\begin{table}[h]
\centering
\begin{tabular}{ccccc}
\toprule
$\lambda$ & ExtFin/GDP & GDP (rel.) & TFP (rel.) & $r$ \\
\midrule
$\infty$ & 1.69 & 1.00 & 1.00 & 4.6\% \\
2.00 & 1.26 & 0.83 & 0.87 & $-2.0\%$ \\
1.75 & 1.06 & 0.81 & 0.86 & $-3.7\%$ \\
1.50 & 0.75 & 0.78 & 0.84 & $-4.0\%$ \\
1.25 & 0.44 & 0.73 & 0.81 & $-4.5\%$ \\
1.00 & 0.00 & 0.68 & 0.78 & $-6.0\%$ \\
\bottomrule
\end{tabular}
\caption{Replication of Figure 2: Long-run effect of financial frictions}
\end{table}

\textbf{Key findings}:
\begin{itemize}
    \item Financial autarky ($\lambda = 1$) reduces GDP by $\sim 32\%$ relative to perfect credit
    \item TFP loss from misallocation: $\sim 22\%$
    \item Interest rates decline with tighter frictions (excess savings due to precautionary motives)
\end{itemize}

%===============================================================================
\bibliographystyle{apalike}
\begin{thebibliography}{9}

\bibitem{buera2013financial}
Buera, Francisco J., and Yongseok Shin.
\textit{Financial frictions and the persistence of history: A quantitative exploration}.
Journal of Political Economy 121.2 (2013): 221-272.

\end{thebibliography}

\end{document}
